% Options for packages loaded elsewhere
\PassOptionsToPackage{unicode}{hyperref}
\PassOptionsToPackage{hyphens}{url}
%
\documentclass[
]{article}
\usepackage{amsmath,amssymb}
\usepackage{lmodern}
\usepackage{iftex}
\ifPDFTeX
  \usepackage[T1]{fontenc}
  \usepackage[utf8]{inputenc}
  \usepackage{textcomp} % provide euro and other symbols
\else % if luatex or xetex
  \usepackage{unicode-math}
  \defaultfontfeatures{Scale=MatchLowercase}
  \defaultfontfeatures[\rmfamily]{Ligatures=TeX,Scale=1}
\fi
% Use upquote if available, for straight quotes in verbatim environments
\IfFileExists{upquote.sty}{\usepackage{upquote}}{}
\IfFileExists{microtype.sty}{% use microtype if available
  \usepackage[]{microtype}
  \UseMicrotypeSet[protrusion]{basicmath} % disable protrusion for tt fonts
}{}
\makeatletter
\@ifundefined{KOMAClassName}{% if non-KOMA class
  \IfFileExists{parskip.sty}{%
    \usepackage{parskip}
  }{% else
    \setlength{\parindent}{0pt}
    \setlength{\parskip}{6pt plus 2pt minus 1pt}}
}{% if KOMA class
  \KOMAoptions{parskip=half}}
\makeatother
\usepackage{xcolor}
\IfFileExists{xurl.sty}{\usepackage{xurl}}{} % add URL line breaks if available
\IfFileExists{bookmark.sty}{\usepackage{bookmark}}{\usepackage{hyperref}}
\hypersetup{
  pdftitle={Final Assignment: Activity Prediction},
  pdfauthor={Sefran12},
  hidelinks,
  pdfcreator={LaTeX via pandoc}}
\urlstyle{same} % disable monospaced font for URLs
\usepackage[margin=1in]{geometry}
\usepackage{color}
\usepackage{fancyvrb}
\newcommand{\VerbBar}{|}
\newcommand{\VERB}{\Verb[commandchars=\\\{\}]}
\DefineVerbatimEnvironment{Highlighting}{Verbatim}{commandchars=\\\{\}}
% Add ',fontsize=\small' for more characters per line
\usepackage{framed}
\definecolor{shadecolor}{RGB}{248,248,248}
\newenvironment{Shaded}{\begin{snugshade}}{\end{snugshade}}
\newcommand{\AlertTok}[1]{\textcolor[rgb]{0.94,0.16,0.16}{#1}}
\newcommand{\AnnotationTok}[1]{\textcolor[rgb]{0.56,0.35,0.01}{\textbf{\textit{#1}}}}
\newcommand{\AttributeTok}[1]{\textcolor[rgb]{0.77,0.63,0.00}{#1}}
\newcommand{\BaseNTok}[1]{\textcolor[rgb]{0.00,0.00,0.81}{#1}}
\newcommand{\BuiltInTok}[1]{#1}
\newcommand{\CharTok}[1]{\textcolor[rgb]{0.31,0.60,0.02}{#1}}
\newcommand{\CommentTok}[1]{\textcolor[rgb]{0.56,0.35,0.01}{\textit{#1}}}
\newcommand{\CommentVarTok}[1]{\textcolor[rgb]{0.56,0.35,0.01}{\textbf{\textit{#1}}}}
\newcommand{\ConstantTok}[1]{\textcolor[rgb]{0.00,0.00,0.00}{#1}}
\newcommand{\ControlFlowTok}[1]{\textcolor[rgb]{0.13,0.29,0.53}{\textbf{#1}}}
\newcommand{\DataTypeTok}[1]{\textcolor[rgb]{0.13,0.29,0.53}{#1}}
\newcommand{\DecValTok}[1]{\textcolor[rgb]{0.00,0.00,0.81}{#1}}
\newcommand{\DocumentationTok}[1]{\textcolor[rgb]{0.56,0.35,0.01}{\textbf{\textit{#1}}}}
\newcommand{\ErrorTok}[1]{\textcolor[rgb]{0.64,0.00,0.00}{\textbf{#1}}}
\newcommand{\ExtensionTok}[1]{#1}
\newcommand{\FloatTok}[1]{\textcolor[rgb]{0.00,0.00,0.81}{#1}}
\newcommand{\FunctionTok}[1]{\textcolor[rgb]{0.00,0.00,0.00}{#1}}
\newcommand{\ImportTok}[1]{#1}
\newcommand{\InformationTok}[1]{\textcolor[rgb]{0.56,0.35,0.01}{\textbf{\textit{#1}}}}
\newcommand{\KeywordTok}[1]{\textcolor[rgb]{0.13,0.29,0.53}{\textbf{#1}}}
\newcommand{\NormalTok}[1]{#1}
\newcommand{\OperatorTok}[1]{\textcolor[rgb]{0.81,0.36,0.00}{\textbf{#1}}}
\newcommand{\OtherTok}[1]{\textcolor[rgb]{0.56,0.35,0.01}{#1}}
\newcommand{\PreprocessorTok}[1]{\textcolor[rgb]{0.56,0.35,0.01}{\textit{#1}}}
\newcommand{\RegionMarkerTok}[1]{#1}
\newcommand{\SpecialCharTok}[1]{\textcolor[rgb]{0.00,0.00,0.00}{#1}}
\newcommand{\SpecialStringTok}[1]{\textcolor[rgb]{0.31,0.60,0.02}{#1}}
\newcommand{\StringTok}[1]{\textcolor[rgb]{0.31,0.60,0.02}{#1}}
\newcommand{\VariableTok}[1]{\textcolor[rgb]{0.00,0.00,0.00}{#1}}
\newcommand{\VerbatimStringTok}[1]{\textcolor[rgb]{0.31,0.60,0.02}{#1}}
\newcommand{\WarningTok}[1]{\textcolor[rgb]{0.56,0.35,0.01}{\textbf{\textit{#1}}}}
\usepackage{graphicx}
\makeatletter
\def\maxwidth{\ifdim\Gin@nat@width>\linewidth\linewidth\else\Gin@nat@width\fi}
\def\maxheight{\ifdim\Gin@nat@height>\textheight\textheight\else\Gin@nat@height\fi}
\makeatother
% Scale images if necessary, so that they will not overflow the page
% margins by default, and it is still possible to overwrite the defaults
% using explicit options in \includegraphics[width, height, ...]{}
\setkeys{Gin}{width=\maxwidth,height=\maxheight,keepaspectratio}
% Set default figure placement to htbp
\makeatletter
\def\fps@figure{htbp}
\makeatother
\setlength{\emergencystretch}{3em} % prevent overfull lines
\providecommand{\tightlist}{%
  \setlength{\itemsep}{0pt}\setlength{\parskip}{0pt}}
\setcounter{secnumdepth}{-\maxdimen} % remove section numbering
\ifLuaTeX
  \usepackage{selnolig}  % disable illegal ligatures
\fi

\title{Final Assignment: Activity Prediction}
\author{Sefran12}
\date{}

\begin{document}
\maketitle

\hypertarget{download-datasets}{%
\subsection{Download datasets}\label{download-datasets}}

\begin{Shaded}
\begin{Highlighting}[]
\FunctionTok{library}\NormalTok{(tidyverse)}
\end{Highlighting}
\end{Shaded}

\begin{verbatim}
## -- Attaching packages --------------------------------------- tidyverse 1.3.1 --
\end{verbatim}

\begin{verbatim}
## v ggplot2 3.3.6     v purrr   0.3.4
## v tibble  3.1.7     v dplyr   1.0.9
## v tidyr   1.2.0     v stringr 1.4.0
## v readr   2.1.2     v forcats 0.5.1
\end{verbatim}

\begin{verbatim}
## -- Conflicts ------------------------------------------ tidyverse_conflicts() --
## x dplyr::filter() masks stats::filter()
## x dplyr::lag()    masks stats::lag()
\end{verbatim}

\begin{Shaded}
\begin{Highlighting}[]
\FunctionTok{library}\NormalTok{(tidymodels)}
\end{Highlighting}
\end{Shaded}

\begin{verbatim}
## -- Attaching packages -------------------------------------- tidymodels 0.2.0 --
\end{verbatim}

\begin{verbatim}
## v broom        0.8.0     v rsample      0.1.1
## v dials        0.1.1     v tune         0.2.0
## v infer        1.0.0     v workflows    0.2.6
## v modeldata    0.1.1     v workflowsets 0.2.1
## v parsnip      0.2.1     v yardstick    0.0.9
## v recipes      0.2.0
\end{verbatim}

\begin{verbatim}
## -- Conflicts ----------------------------------------- tidymodels_conflicts() --
## x scales::discard() masks purrr::discard()
## x dplyr::filter()   masks stats::filter()
## x recipes::fixed()  masks stringr::fixed()
## x dplyr::lag()      masks stats::lag()
## x yardstick::spec() masks readr::spec()
## x recipes::step()   masks stats::step()
## * Search for functions across packages at https://www.tidymodels.org/find/
\end{verbatim}

\begin{Shaded}
\begin{Highlighting}[]
\FunctionTok{library}\NormalTok{(ranger)}

\NormalTok{df }\OtherTok{\textless{}{-}} \FunctionTok{read\_csv}\NormalTok{(}\StringTok{"https://d396qusza40orc.cloudfront.net/predmachlearn/pml{-}training.csv"}\NormalTok{, }\AttributeTok{na =} \FunctionTok{c}\NormalTok{(}\StringTok{"NA"}\NormalTok{, }\StringTok{"\#DIV/0!"}\NormalTok{))}
\end{Highlighting}
\end{Shaded}

\begin{verbatim}
## New names:
## * `` -> `...1`
\end{verbatim}

\begin{verbatim}
## Rows: 19622 Columns: 160
## -- Column specification --------------------------------------------------------
## Delimiter: ","
## chr   (4): user_name, cvtd_timestamp, new_window, classe
## dbl (150): ...1, raw_timestamp_part_1, raw_timestamp_part_2, num_window, rol...
## lgl   (6): kurtosis_yaw_belt, skewness_yaw_belt, kurtosis_yaw_dumbbell, skew...
## 
## i Use `spec()` to retrieve the full column specification for this data.
## i Specify the column types or set `show_col_types = FALSE` to quiet this message.
\end{verbatim}

\begin{Shaded}
\begin{Highlighting}[]
\NormalTok{assignment\_df }\OtherTok{\textless{}{-}} \FunctionTok{read\_csv}\NormalTok{(}\StringTok{"https://d396qusza40orc.cloudfront.net/predmachlearn/pml{-}testing.csv"}\NormalTok{, }\AttributeTok{na =} \FunctionTok{c}\NormalTok{(}\StringTok{"NA"}\NormalTok{, }\StringTok{"\#DIV/0!"}\NormalTok{)) }\CommentTok{\# we can\textquotesingle{}t call it test data because we don\textquotesingle{}t have labels}
\end{Highlighting}
\end{Shaded}

\begin{verbatim}
## New names:
## Rows: 20 Columns: 160
## -- Column specification
## -------------------------------------------------------- Delimiter: "," chr
## (3): user_name, cvtd_timestamp, new_window dbl (57): ...1,
## raw_timestamp_part_1, raw_timestamp_part_2, num_window, rol... lgl (100):
## kurtosis_roll_belt, kurtosis_picth_belt, kurtosis_yaw_belt, skewn...
## i Use `spec()` to retrieve the full column specification for this data. i
## Specify the column types or set `show_col_types = FALSE` to quiet this message.
## * `` -> `...1`
\end{verbatim}

We see that we have a lot of variables, and a reasonably good amount of
observations (19622/160 \textgreater{} 100, the heuristic for usual ML
models on small datasets). Before doing any model development, we need
to keep a test set apart. Also, notice many of the features are riddled
with NAs, and that this dataset has a natural time ordering. I doubt
we'd need to use time-series variables here (the test set does not let
us). We will choose a train/dev/test splitting schema. The size of the
dev and test sets will be chosen as for us to detect 1\% changes in our
metric with 95\% coverage.

Which metric? Let's see class imbalance:

\begin{Shaded}
\begin{Highlighting}[]
\NormalTok{df}\SpecialCharTok{$}\NormalTok{classe }\SpecialCharTok{\%\textgreater{}\%} \FunctionTok{table}\NormalTok{() }\SpecialCharTok{\%\textgreater{}\%} \FunctionTok{prop.table}\NormalTok{()}
\end{Highlighting}
\end{Shaded}

\begin{verbatim}
## .
##         A         B         C         D         E 
## 0.2843747 0.1935073 0.1743961 0.1638977 0.1838243
\end{verbatim}

Some class imbalance. Probably a micro-averaged F1 score is a reasonable
initial choice here. So approx 1000 examples will be enough probably.

\hypertarget{train-dev-test-split}{%
\subsection{Train-dev-test split}\label{train-dev-test-split}}

\begin{Shaded}
\begin{Highlighting}[]
\CommentTok{\# train test split}
\NormalTok{train\_test\_split }\OtherTok{\textless{}{-}} \FunctionTok{initial\_split}\NormalTok{(df, }\AttributeTok{prop =} \DecValTok{1} \SpecialCharTok{{-}} \DecValTok{1000}\SpecialCharTok{/}\FunctionTok{nrow}\NormalTok{(df), }\AttributeTok{strata =}\NormalTok{ classe)}
\NormalTok{train\_dev\_df }\OtherTok{\textless{}{-}} \FunctionTok{training}\NormalTok{(train\_test\_split)}
\NormalTok{test\_df }\OtherTok{\textless{}{-}} \FunctionTok{testing}\NormalTok{(train\_test\_split)}

\CommentTok{\# train into train dev split}
\NormalTok{train\_dev\_split }\OtherTok{\textless{}{-}} \FunctionTok{initial\_split}\NormalTok{(train\_dev\_df, }\AttributeTok{prop =} \DecValTok{1} \SpecialCharTok{{-}} \DecValTok{1000}\SpecialCharTok{/}\FunctionTok{nrow}\NormalTok{(train\_dev\_df), }\AttributeTok{strata =}\NormalTok{ classe)}
\NormalTok{train\_df }\OtherTok{\textless{}{-}} \FunctionTok{training}\NormalTok{(train\_dev\_split)}
\NormalTok{dev\_df }\OtherTok{\textless{}{-}} \FunctionTok{testing}\NormalTok{(train\_dev\_split)}
\end{Highlighting}
\end{Shaded}

Why train/dev/test? It's already well known by now but most high-powered
ML models easily ``overfit'' to the data set used for optimizing choices
of modeling. So we, when we can, need a consistent (sometimes unbiased)
estimate of performance of the final model. Our dev set will be the one
used to make modeling choices, and the final test set will be used just
once, at the end, to estimate consistent measures of performance.

\hypertarget{feature-engineering}{%
\subsection{Feature engineering}\label{feature-engineering}}

Let's do some feature engineering. For simplicity, I'll do these
transformations: - NA's will be made categories - Columns with too many
NA's, NA's will be filled with median values and an indicator variable
will be made

And we will start with a simple ranger random forest. If you look at the
page for the dataset, you'll see that they had more or less 165000 data
points and have achieved 99.41\% accuracy (macro average) and 0.994 F1
score (micro average). I don't know if this was on testing, or just in
training (probably training. The size of the dataset is too big for it
to be testing). So we will probably need to aim, for a dataset of
something less than 20000 instances, and proper testing on testing set,
between 0.95 and 0.98 F1 score?

\begin{Shaded}
\begin{Highlighting}[]
\NormalTok{rf\_recipe }\OtherTok{\textless{}{-}}\NormalTok{ train\_df }\SpecialCharTok{\%\textgreater{}\%}
    \FunctionTok{recipe}\NormalTok{(classe }\SpecialCharTok{\textasciitilde{}}\NormalTok{ .) }\SpecialCharTok{\%\textgreater{}\%} 
    \FunctionTok{add\_role}\NormalTok{(}\StringTok{\textasciigrave{}}\AttributeTok{...1}\StringTok{\textasciigrave{}}\SpecialCharTok{:}\NormalTok{num\_window, }\AttributeTok{new\_role =} \StringTok{"id variable"}\NormalTok{) }\SpecialCharTok{\%\textgreater{}\%} 
    \FunctionTok{step\_indicate\_na}\NormalTok{(}\FunctionTok{all\_predictors}\NormalTok{()) }\SpecialCharTok{\%\textgreater{}\%} 
    \FunctionTok{step\_impute\_median}\NormalTok{(}\FunctionTok{all\_numeric\_predictors}\NormalTok{()) }\SpecialCharTok{\%\textgreater{}\%} 
    \FunctionTok{step\_novel}\NormalTok{(}\FunctionTok{all\_nominal\_predictors}\NormalTok{()) }\SpecialCharTok{\%\textgreater{}\%} 
    \FunctionTok{step\_unknown}\NormalTok{(}\FunctionTok{all\_nominal\_predictors}\NormalTok{())}

\NormalTok{trained\_rf\_recipe }\OtherTok{\textless{}{-}}\NormalTok{ rf\_recipe }\SpecialCharTok{\%\textgreater{}\%} \FunctionTok{prep}\NormalTok{()}
\end{Highlighting}
\end{Shaded}

\begin{verbatim}
## New names:
## * `na_ind_...1` -> `na_ind_`
\end{verbatim}

\begin{Shaded}
\begin{Highlighting}[]
\NormalTok{train\_data }\OtherTok{\textless{}{-}} \FunctionTok{bake}\NormalTok{(trained\_rf\_recipe, }\AttributeTok{new\_data =}\NormalTok{ train\_df)}
\end{Highlighting}
\end{Shaded}

\begin{verbatim}
## New names:
## * `na_ind_...1` -> `na_ind_`
\end{verbatim}

\begin{Shaded}
\begin{Highlighting}[]
\NormalTok{dev\_data }\OtherTok{\textless{}{-}} \FunctionTok{bake}\NormalTok{(trained\_rf\_recipe, }\AttributeTok{new\_data =}\NormalTok{ dev\_df)}
\end{Highlighting}
\end{Shaded}

\begin{verbatim}
## New names:
## * `na_ind_...1` -> `na_ind_`
\end{verbatim}

\end{document}
